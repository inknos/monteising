\documentclass[11]{article}

\begin{document}

\newcommand\tab[1][1cm]{\hspace*{#1}}

\section*{Lattice Description}

Being $N$ the number of spins along a one-dimensional edge of the cubic Lattice, \\ 
the “lattice” array $\mathbf{L}$ of a D-dimensional cubic Lattice is made up by $N^{D}$ boolean entries.\\ 
$$\mathbf{L} = (L_0,L_1,..,L_{N^D-1})$$
Let $i$ be the index of the array, ranging from $0$ to $N^D-1$.\\
Let $i$-$spin$ denote the spin represented by the $L_i$ boolean entry. \\
From index $i$ the cartesian coordinates of the $i$-spin can be computed: \\ 
$$\mathbf{a} = (a_0,a_1,..,a_{D-1})$$

This can be done if a convention is set, which in our case study is the following: \\ 

\begin{itemize}
	\item The origin of the D-dimensional space is chosen to be a vertex of the lattice cube (the vertex is a spin itself) such that each edge starting from \textit{O} is aligned along one of the D positive directions of the space.
	\item The distance between each couple of adjacent spins is set to one. 
	\item The spins are enumerated starting from increasing the first dimension’s coordinate, than the second and so forth.
\end{itemize}

%figura caso 1 , 2 , 3 dimensionale
\textbf{N.B.} from this rules we can observe that, given a specific spin, each coordinate $a_j$ can take values in range 
$\{0,1,…,N-1\}$. \\ 

Let’s now consider different cases starting from the simpler \textit{D=1} case to the \textit{D=3} case, in order to understand and generalize the coordinates’ computation. \\
Before starting we introduce two useful operators for integer numbers: 
\begin{itemize}
	\item / : Division Operator	
		\begin{itemize}
			\item[-] a/n returns the quotient of the euclidean division of a by n
			\item[e.g.] : $6/3 = 2$ ; $9/4 = 2$
		\end{itemize}
	\item \% : Modulo Operator
		\begin{itemize}
			\item[-] a\%n returns the remainder of the euclidean division of a by n 
			\item[e.g.] : $6\%3 = 0$ ; $9\%4 = 1$
		\end{itemize} 
\end{itemize}

\textbf{D=1}
A 1-dimensional lattice is made by $N$ spins aligned along a single direction which we call $x$.\\
In this case the index $i$ represents itself the $x$ coordinate of the spin.\\

\textbf{D=2}
A 2-dimensional Lattice is a square made by $N^2$ spins. Let’s introduce the second direction $y$.\\
The index can be written in the following form : \\
$i = a_0 + a_1N$ 

$a_1$ can be seen as the number of $N$-spin lines which are stacked starting from the quote $y=0$,this number is the $y$ cordinate itself while the number or remaining spin $a_0$ is the $x$ coordinate. \\
$a_1$ is computed from $i$ divided by $N$ while $a_0$ is computed from $i modulo N$.

So the $(x,y)$ coordinates of the $i$-spin are given by : $$(a_0,a_1)=(i\%N ; i/N)$$ \\


\textbf{D=3}

A 3-dimensional Lattice is a cube made by $N^{3}$ spins. Let’s introduce the third direction $z$ \\  
The index $i$ can be written in the following form :\\  
$i = a_0 + a_1N + a_2N^2$

$a_2$ can be seen as the number of $N^2$-spin squares which are vertically stacked starting from the plane $z=0$, this number is 
the $z$ coordinate itself. $a_2$ is computed from $i divided by N$.
Once we have $a_2$ the remaining $(x,y)$ coordinates can be computed from $i\%N^2 = a_0 + a_1N$ like in case \textbf{D=2}

So the $(x,y,z)$ coordinates of the $i$-spin are given by : \\
$$(a_0,a_1,a_2)=\left( (i\%N^2)\%N , (i\%N^2)/N , i/N^2 \right)$$


We now generalize to a $D$-dimensional lattice, the index $i$ can be expressed as :
$$ i = \sum_{j=1}^{D-1}a_jN^j $$
Let's compute the $\{a_j\}$ coordinates starting from $j=D-1$ and establishing a recurrence relationship :
\begin{eqnarray*}	
	a_{D-1} &=& i/N^{D-1}  \\
	a_{D-2} &=& [i\% N^{D-1}]/N^{D-2} \\
			&.\\
			&.\\
	a_j &=& [i\%N^{j+1}]/N^j	
\end{eqnarray*}

\textbf{N.B.} for $j=0$ we have : $a_0 = i\%N$	

According to this algorithm we are able to compute the coordinates of the $i$-spin


 
\section*{Energy of the Lattice} 

The Energy of the Lattice is computed according to the following formula : 
$$H = \sum_{<l,m>}^{} -J\sigma_i\sigma_j $$

\tab J is a constant interaction term between each couple <l,m> of spins .\\
\tab The sum is performed over all possible couples <l,m> of adjacent spins.\\

The Lattice class provides a method wich returns the energy of the system. \\
The method's implemented algorithm takes the lattice array and performs two for loops. \\
\begin{enumerate}
\item[i)] \textbf{First Loop} : over the array's index $i = (0,1,..,N^D-1)$ 
	\begin{enumerate}
	\item[d)] \textbf{Second Loop} : over the dimension of the lattice $d = (0,1,..,D-1)$: 
	\end{enumerate}	
\end{enumerate}


The key idea is that, for each fixed $i$-spin, along each dimension $d$ we can find two neighbours $i_{d_\pm}$-spin which are defined respectively as the neighbour of the $i$-spin found on the increasing and decreasing d-coordinate.   

%figura caso 1d , 2d , 3d 
For each first loop iteration, $2D$ interacting neighbours are found. By summing up their energy interaction terms we obtain the contribute of the $i$-spin to the total energy of the system. \\
So the energy can be rewritten as : 

$$H = \frac{1}{2}\sum_{i=0}^{N^D-1}\sum_{d=0}^{D-1}\sum_{\pm}^{} -J\sigma_i\sigma_{i_{d_\pm}}$$ 

It's easy to observe that, performing the first summation $\sum_{i}^{}$ , double countings of couples occur and a factor $1/2$ is required. \\ 
For each fixed index $i$ ,the algorithm takes into account only the interaction term with the 
$i_{d_+}$-spin so that the number of operation is halved. So the algorithm performs the summation :
  
$$H = \sum_{i=0}^{N^D-1}\sum_{d=0}^{D-1}-J(L_i \ \hat{} \ L_{i_{d_+}})$$ 

%da mettere a piè di pagina VVVVVVV
$sigma_i\sigma_{i_{d_+}}$ can take values $\pm1$ wether the two spins are aligned or not. Since spins are represented by boolean entries of array $\mathbf{L}$, the XOR bitwise operator ^ is applied on couple $(L_i,L_{i_{d_+}})$ so that it returns the same value of $sigma_i\sigma_{i_{d_+}}$   
%da mettere a piè di pagina ^^^^^^^
      




\end{document}

